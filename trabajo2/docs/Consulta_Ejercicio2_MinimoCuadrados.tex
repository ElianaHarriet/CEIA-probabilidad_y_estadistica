\documentclass[11pt,a4paper]{article}
\usepackage[utf8]{inputenc}
\usepackage[spanish]{babel}
\usepackage{geometry}
\usepackage{graphicx}
\usepackage{amsmath}
\usepackage{amsfonts}
\usepackage{amssymb}
\usepackage{fancyhdr}
\usepackage{titling}
\usepackage{xcolor}

\geometry{margin=2.5cm}

\definecolor{ubaazul}{RGB}{0,47,95}
\definecolor{ubagris}{RGB}{88,88,90}

\begin{document}

\section{Ejercicio 2}
\textbf{Estimadores de mínimos cuadrados}

\subsection{Enunciado}
Se pretende estimar los valores de producción $Y$ (en miles de toneladas) de cierto material, en función del tiempo transcurrido $X$ (en meses) usando los valores de la tabla:

\begin{center}
\begin{tabular}{|c|c|c|c|c|c|}
\hline
$X$ & 2 & 6 & 14 & 15 & 23 \\
\hline
$Y$ & 3 & 48 & 160 & 201 & 424 \\
\hline
\end{tabular}
\end{center}

Se plantea un modelo de la forma $Y = a + bx + cx^2$. Encontrar los estimadores de mínimos cuadrados para $a$, $b$ y $c$ en este modelo.

\subsection{Solución}

En este ejercicio aplicaremos el método de mínimos cuadrados para estimar los parámetros de un modelo de regresión cuadrática. El objetivo es minimizar la suma de cuadrados de los residuos.

\textbf{Paso 1: Formulación del problema}

Tenemos el modelo cuadrático:
\[
Y = a + bx + cx^2 + \varepsilon
\]

donde $\varepsilon$ representa el error aleatorio. Los datos observados son:
\begin{center}
\begin{tabular}{|c|c|c|c|c|c|}
\hline
$x_i$ & 2 & 6 & 14 & 15 & 23 \\
\hline
$y_i$ & 3 & 48 & 160 & 201 & 424 \\
\hline
\end{tabular}
\end{center}

\textbf{Paso 2: Conversión a forma matricial}

El modelo cuadrático se puede expresar como un modelo de regresión lineal múltiple:
\[
\mathbf{y} = \mathbf{X}\boldsymbol{\beta} + \boldsymbol{\varepsilon}
\]

donde:
\begin{itemize}
    \item $\mathbf{y} = \begin{bmatrix} 3 \\ 48 \\ 160 \\ 201 \\ 424 \end{bmatrix}$ es el vector de respuestas
    \item $\boldsymbol{\beta} = \begin{bmatrix} a \\ b \\ c \end{bmatrix}$ es el vector de parámetros a estimar
    \item $\mathbf{X}$ es la matriz de diseño
\end{itemize}

La matriz de diseño $\mathbf{X}$ se construye con las columnas $[1, x, x^2]$:
\[
\mathbf{X} = \begin{bmatrix}
1 & 2 & 4 \\
1 & 6 & 36 \\
1 & 14 & 196 \\
1 & 15 & 225 \\
1 & 23 & 529
\end{bmatrix}
\]

\textbf{Paso 3: Fórmula de mínimos cuadrados}

El estimador de mínimos cuadrados está dado por:
\[
\hat{\boldsymbol{\beta}} = (\mathbf{X}^T\mathbf{X})^{-1}\mathbf{X}^T\mathbf{y}
\]

\textbf{Paso 4: Cálculo de $\mathbf{X}^T\mathbf{X}$}

\[
\mathbf{X}^T\mathbf{X} = \begin{bmatrix}
    1 & 1 & 1 & 1 & 1 \\
    2 & 6 & 14 & 15 & 23 \\
    4 & 36 & 196 & 225 & 529
\end{bmatrix}
\begin{bmatrix}
    1 & 2 & 4 \\
    1 & 6 & 36 \\
    1 & 14 & 196 \\
    1 & 15 & 225 \\
    1 & 23 & 529
\end{bmatrix}
\]

\[
\mathbf{X}^T\mathbf{X} = \begin{bmatrix}
5 & 60 & 990 \\
60 & 990 & 18510 \\
990 & 18510 & 370194
\end{bmatrix}
\]

\textbf{Paso 5: Cálculo de $\mathbf{X}^T\mathbf{y}$}

\[
\mathbf{X}^T\mathbf{y} = \begin{bmatrix}
    1 & 1 & 1 & 1 & 1 \\
    2 & 6 & 14 & 15 & 23 \\
    4 & 36 & 196 & 225 & 529
\end{bmatrix}
\begin{bmatrix}
    3 \\
    48 \\
    160 \\
    201 \\
    424
\end{bmatrix}
\]

\[
\mathbf{X}^T\mathbf{y} = \begin{bmatrix} 836 \\ 15301 \\ 302621 \end{bmatrix}
\]

\textbf{Paso 6: Cálculo de $(\mathbf{X}^T\mathbf{X})^{-1}$}

\[
\mathbf{X}^T\mathbf{X} = \begin{bmatrix} 5 & 60 & 990 \\ 60 & 990 & 18510 \\ 990 & 18510 & 370194 \end{bmatrix}
\]

\[
(\mathbf{X}^T\mathbf{X})^{-1} \approx \begin{bmatrix}
1.555 & -0.253 & 0.0085 \\
-0.253 & 0.0567 & -0.00216 \\
0.0085 & -0.00216 & 0.0000879
\end{bmatrix}
\]

\textbf{Paso 7: Cálculo final de los estimadores}

\begin{align*}
\hat{\boldsymbol{\beta}} &= (\mathbf{X}^T\mathbf{X})^{-1}\mathbf{X}^T\mathbf{y} \\
&\approx \begin{bmatrix} 1.555 & -0.253 & 0.0085 \\ -0.253 & 0.0567 & -0.00216 \\ 0.0085 & -0.00216 & 0.0000879 \end{bmatrix} \begin{bmatrix} 836 \\ 15301 \\ 302621 \end{bmatrix} \\
&\approx \begin{bmatrix} -1.434 \\ 2.865 \\ 0.678 \end{bmatrix}
\end{align*}



\textbf{Conclusión}

Los estimadores de mínimos cuadrados para el modelo $Y = a + bx + cx^2$ son:

\[
\boxed{
\begin{aligned}
\hat{a} &= -1.434 \\
\hat{b} &= 2.865 \\
\hat{c} &= 0.678
\end{aligned}
}
\]

El modelo estimado es:
\[
\hat{Y} = -1.434 + 2.865x + 0.678x^2
\]

Este modelo cuadrático captura la relación no lineal entre el tiempo transcurrido y la producción del material.

\end{document}
